%Latex Template created by Sanjay Shitole 
\documentclass[12pt,a4paper,final,oneside]{report}

\usepackage{geometry}
\usepackage{amsfonts}
\usepackage{amssymb}
\usepackage{graphics}
\usepackage{graphicx}
\usepackage{amsmath}
\usepackage{array}
\usepackage[pdftex]{hyperref}
\usepackage{epstopdf}

\begin{document}

\chapter{Introduction}
You have probably heard for years about the revolution the Internet would bring to teaching and learning. As with so many promises of revolution, the changes havent materialized. Instead, a new suite of tools, called course management systems , can be used to enhance your teaching by taking advantage of the Internet without replacing the need for a teacher.

\section{What is a CMS}
CMSs are web applications, meaning they run on a server and are accessed by using a web browser. The server is usually located in your university or department, but it can be anywhere in the world. CMSs give educators tools to create a course web site and provide access control so only enrolled students can view it. Aside from access control, CMSs o er a wide variety of tools that can make your course more e ective. They provide an easy way to upload and share materials, hold online discussions and chats, give quizzes and surveys, gather and review assignments, and record grades. Lets take a quick look at each of these features and how they might be useful:

\subsection{Uploading and sharing materials}
Most CMSs provide tools to easily publish content. Instead of using an HTML editor and then sending your documents to a server via FTP, you simply use a web form to store your syllabus on the server. Many instructors upload their syllabus, lecture notes, reading assignments, and articles for students to access whenever they want. 


\subsection{Forums and chats}
Online forums and chats provide a means of communication outside of classroom meetings. Forums give your students more time to generate their responses and can lead to more thoughtful discussions. Chats, on the other hand, give you a way to quickly and easily communicate with remote students. They can be used for everything from course announcements to entire lectures.


\subsection{Quizzes and surveys}

Online quizzes and surveys can be graded instantaneously. They are a great tool for giving students rapid feedback on their performance and for gauging their comprehension of materials. Many publishers now provide banks of test questions tied to book chapters.

\section{Gathering and reviewing assignments}

Online assignment submissions are an easy way to track and grade student assign-ments. Also, research indicates that using an online environment for anonymous student peer reviews of each others work increases student motivation and perfor-mance.

\subsection{Recording grades}

An online grade book can give your students up-to-date information about their performance in your course. Online grades can also help you comply with new privacy rules that prohibit posting grades with personal identi ers in public places. CMS grade books allow students to see only their own grades, never another students.

A CMS combines all of these features into one integrated package. Once youve learned how to use a CMS, youll be free to concentrate on teaching and learning instead of writing and maintaining your own software.


\section{Why Should You Use a CMS}

Chalk and talk is still the predominant method of delivering instruction. While traditional face-to-face meetings can still be e ective, applying the tools listed above opens up new possibilities for learning that werent possible just a few years ago. Currently, there is a lot of research into how to e ectively combine online learning and face-to-face meetings in what are called hybrid courses. There are a number of other reasons to think about using a CMS in your courses:


\subsection{Student demand}
Students are becoming more technically savvy, and they want to get many of their course materials o the Web. Once online, they can access the latest information at any time and can make as many copies of the materials as they need.

\subsection{Recording grades}
With rising tuitions, many students are working more hours to make ends meet while they are in school. About half of all students now work at least 20 hours a week to meet school expenses. With a CMS, they can communicate with you or their peers whenever their schedules permit. They can also take quizzes or read course material during their lunch break.

\subsection{Better courses}
If used well, CMSs can make your classes more e ective and e cient. By mov-ing some parts of your course online, you can more e ectively take advantage of scheduled face-to-face time to engage students questions and ideas. For example, if you move your content delivery from an in-class lecture to an online document, you can then use lecture time to ask students about what they didnt understand. If you also use an online forum, you can bring the best ideas and questions from the forum into your classroom.


\chapter{Moodle}

Moodle is a software package for producing Internet-based courses and web sites. It is a global development project designed to support a social constructionist framework of education. Moodle is provided freely as Open Source software (under the GNU Public License). Basically this means Moodle is copyrighted, but that you have additional freedoms. You are allowed to copy, use and modify Moodle provided that you agree to: provide the source to others; not modify or remove the original license and copyrights, and apply this same license to any derivative work. Moodle can be installed on any computer that can run PHP, and can support an SQL type database (for example MySQL). It can be run on Windows and Mac operating systems and many avors of linux (for example Red Hat or Debian GNU). The word Moodle was originally an acronym for Modular Object-Oriented Dynamic Learning Environment, which is mostly useful to programmers and education theorists.


\section{BACKGROUND}
Moodle is an active and evolving work in progress. Development was started by Martin Dougiamas who continues to lead the project. A number of early proto-types were produced and discarded before he released version 1.0 upon a largely unsuspecting world on August 20, 2002. This version was targeted towards smaller, more intimate classes at University level, and was the subject of research case stud-ies that closely analysed the nature of collaboration and re ection that occurred among these small groups of adult participants. Since then there has been steady series of new releases adding new features, better scalability and improved perfor-mance. As Moodle has spread and the community has grown, more input is being drawn from a wider variety of people in di erent teaching situations. A growing number of people from around the world are contributing to Moodle in di erent 





\end{document}

